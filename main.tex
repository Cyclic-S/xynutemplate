\documentclass[UTF8,heading=true,12pt,font=songti]{article}
\usepackage{ctex}
\usepackage{cite}
\usepackage{fontawesome}
\usepackage{pdfpages}
\usepackage{titlesec}
\usepackage{xparse,amsmath,amsfonts,mathrsfs,amsthm,amssymb,tikz,bm,graphicx,url,listings,subfig,float,subcaption}
\usepackage{physics}
%\usepackage{mathptmx}% 新罗马字体
\usepackage[ruled,linesnumbered]{algorithm2e}
\usepackage[colorlinks,linkcolor=blue]{hyperref} % black 
\usepackage{gbt7714}% 参考文献格式
\usepackage[left=3cm,right=2.5cm,top=2.5cm,bottom=2.5cm]{geometry}% ,showframe 显示框图
\usepackage{lipsum}% 生成示例
\usepackage{pset}% 论文格式设置文件

%封面信息
\college{数学与统计学院}%学院
\serialnumber{20215034000}%学号
\major{信息与计算科学}%专业班级
\advisor{NAME}%指导老师
\grade{2021级} %届次
\submityear{25}%提交年份
\submitmonth{1}%提交月份
\entitle{Title}		%英文标题
\entitletra{}     %英文多余一行
\author{NAME}   %作者
\submitdate{1}   %提交日
\titlefirst{标题}      %题目第一行
\titlesecond{Subtitle}  %题目第二行
\advisors{Pr.}	 %指导老师带职称

\begin{document}
	\maketitle% 标题页
	\newpage
	\tableofcontents% 生成目录
	\setcounter{page}{0}% 此为第零页
	\thispagestyle{empty}% 页眉页脚为空
	\newpage
	\begin{center}
		\songti \zihao{3}\bfseries{\the\titlefirst}%中文标题
	\end{center}
	
	\begin{table}[H]
		\vspace{-1.5em}
		\centering
		\begin{spacing}{1.5}
			\songti \zihao{5}
			\begin{tabular}{rl} %设置表格字体位置(右左)
				学生姓名: {\the\author} & 学号: {\the\serialnumber}\\ 
				{\the\college} & {\the\major}专业\\
				指导教师: {\the\advisor} & 职称: {\the\advisors}\\	
			\end{tabular}
		\end{spacing}
		\vspace{-1.5em}
	\end{table}
	
	\addcontentsline{toc}{section}{摘\quad 要}\tolerance=500 % 将摘要放进目录
	\noindent {\heiti \zihao{-4}\bfseries{摘\quad 要}}: \kaishu\zihao{-4} \ 摘要内容 % 无缩进
	
	
	\noindent {\heiti \zihao{-4}\bfseries{关键词}}:\kaishu\zihao{-4} \ 内容 % 无缩进
	
	\begin{center}
		\zihao{3}{\the\entitle}%英文标题
	\end{center}
	\addcontentsline{toc}{section}{\bfseries Abstract}\tolerance=500 % 将Abstract放进目录
	\noindent {\zihao{-4}\bfseries{Abstract}}:\ \zihao{-4} \lipsum[1]
	
	\noindent {\zihao{-4}\bfseries{Keywords}}:\ \zihao{-4} Ab
	
	%\onehalfspacing %1.5倍行间距
    \section{前言}
    \songti
	$$
    f(z_0)=\frac{1}{2\pi i}\oint_{\Gamma}\frac{f(z)}{z-z_0}\dd z.
    $$

    \section{第一节}
    \lipsum[3]
    \begin{table}[H]
        \centering
        \caption{符号说明}
        \begin{tabular}{clc} %设置表格字体位置(中左中)
            \hline
            \textbf{符号} & \textbf{说明} & \textbf{单位} \\ \hline
            $\eta$ & 定日镜的光学效率 & 无量纲\\	
            \hline
        \end{tabular}
    \end{table}

    \subsection{第一小节}
    引入参考文献\cite{bazlov2024momentsrepresentationnumbers}


    \subsubsection{小小一节}
    \begin{figure}[H]
		\centering
		\subfloat[示意图]{\includegraphics[width = 0.5\linewidth]{figure/logo-r.eps}}
		\hfill
		\subfloat[示意图]{\includegraphics[width = 0.5\linewidth]{figure/logo-r.eps}}
		\hfill
		\subfloat[示意图]{\includegraphics[width = 0.5\linewidth]{figure/logo-r.eps}}
		\hfill
		\subfloat[示意图]{\includegraphics[width = 0.5\linewidth]{figure/logo-r.eps}}
		\caption{示意图}
		\label{fig:label2}
	\end{figure}  

    \bibliography{ref/ref}
    \addcontentsline{toc}{section}{参考文献}\tolerance=500 % 将参考文献放进目录

\end{document}
    
